\documentclass[12pt,a4paper,oneside]{report}


\usepackage{amsmath,amssymb,calc,ifthen}

\usepackage{float}
%\usepackage{cancel}

\usepackage[table,usenames,dvipsnames]{xcolor} % for coloured cells in tables

\usepackage{tikz}

% Allows us to click on links and references!

\usepackage{hyperref}
\hypersetup{
    colorlinks,
    citecolor=black,
    filecolor=black,
    linkcolor=black,
    urlcolor=black
}

% Nice package for plotting graphs
% See excellent guide:
% http://www.tug.org/TUGboat/tb31-1/tb97wright-pgfplots.pdf
\usetikzlibrary{plotmarks}
\usepackage{amsmath,graphicx}
\usepackage{epstopdf}
\usepackage{caption}
\usepackage{subcaption}

% highlight - useful for TODOs and similar
\usepackage{color}
\newcommand{\hilight}[1]{\colorbox{yellow}{#1}}

\newcommand\ci{\perp\!\!\!\perp} % perpendicular sign
\newcommand*\rfrac[2]{{}^{#1}\!/_{#2}} % diagonal fraction

\usepackage{listings}



% margin size
\usepackage[margin=2.5cm]{geometry}

\tikzstyle{state}=[circle,thick,draw=black, align=center, minimum size=2.1cm,
inner sep=0]
\tikzstyle{vertex}=[circle,thick,draw=black]
\tikzstyle{terminal}=[rectangle,thick,draw=black]
\tikzstyle{edge} = [draw,thick]
\tikzstyle{lo} = [edge,dotted]
\tikzstyle{hi} = [edge]
\tikzstyle{trans} = [edge,->]


\definecolor{mygreen}{rgb}{0,0.6,0}
\definecolor{mygray}{rgb}{0.5,0.5,0.5}
\definecolor{mymauve}{rgb}{0.58,0,0.82}

\lstset{ %
  backgroundcolor=\color{white},   % choose the background color; you must add 
%\usepackage{color} or \usepackage{xcolor}
  basicstyle=\footnotesize,        % the size of the fonts that are used for the 
%code
  breakatwhitespace=false,         % sets if automatic breaks should only happen 
%at whitespace
  breaklines=true,                 % sets automatic line breaking
  captionpos=b,                    % sets the caption-position to bottom
  commentstyle=\color{mygreen},    % comment style
  deletekeywords={...},            % if you want to delete keywords from the 
%given language
  escapeinside={\%*}{*)},          % if you want to add LaTeX within your code
  extendedchars=true,              % lets you use non-ASCII characters; for 
%8-bits encodings only, does not work with UTF-8
  frame=single,                    % adds a frame around the code
  keepspaces=true,                 % keeps spaces in text, useful for keeping 
%indentation of code (possibly needs columns=flexible)
  keywordstyle=\color{blue},       % keyword style
  language=Octave,                 % the language of the code
  morekeywords={*,...},            % if you want to add more keywords to the set
  numbers=left,                    % where to put the line-numbers; possible 
%values are (none, left, right)
  numbersep=5pt,                   % how far the line-numbers are from the code
  numberstyle=\tiny\color{mygray}, % the style that is used for the line-numbers
  rulecolor=\color{black},         % if not set, the frame-color may be changed 
%on line-breaks within not-black text (e.g. comments (green here))
  showspaces=false,                % show spaces everywhere adding particular 
%underscores; it overrides 'showstringspaces'
  showstringspaces=false,          % underline spaces within strings only
  showtabs=false,                  % show tabs within strings adding particular 
%underscores
  stepnumber=2,                    % the step between two line-numbers. If it's 
%1, each line will be numbered
  stringstyle=\color{mymauve},     % string literal style
  tabsize=2,                       % sets default tabsize to 2 spaces
  title=\lstname                   % show the filename of files included with 
%\lstinputlisting; also try caption instead of title
}

%\title{EPSRC Centre for Doctoral Training in Medical Imaging\\\hfill\\MRes Project Plan}
% \title{Differential Diagnosis of Alzheimer subtypes through disease progression modelling}
% \author{
% Razvan Valentin Marinescu\\
% Student Number: 14060166\\
% \texttt{razvan.marinescu.14@ucl.ac.uk}
% }

\begin{document}
\belowdisplayskip=12pt plus 3pt minus 9pt
\belowdisplayshortskip=7pt plus 3pt minus 4pt


\begin{titlepage}
\begin{center}

% Upper part of the page. The '~' is needed because \\
% only works if a paragraph has started.
\includegraphics[width=0.2\textwidth]{ucl-logo2}~\\[1cm]

\textsc{\LARGE University College London}\\[1.5cm]

\textsc{\Large MRes Project Plan}\\[0.5cm]

% Title
\HRule \\[0.4cm]
{ \Large Differential Diagnosis of Alzheimer subtypes through disease progression modelling \\[0.4cm] }

\HRule \\[1.5cm]

% Author and supervisor
\begin{minipage}{0.4\textwidth}
\begin{flushleft} \large
\emph{Author:}\\
R\u{a}zvan Valentin \textsc{Marinescu}
\end{flushleft}
\end{minipage}
\begin{minipage}{0.4\textwidth}
\begin{flushright} \large
\emph{Supervisors:} \\
Prof. Daniel \textsc{Alexander}\\
Dr. Sebastian \textsc{Crutch}
\end{flushright}
\end{minipage}

\vfill

EPSRC Centre for Doctoral Training in Medical Imaging\\ University College London

\vfill

% Bottom of the page
{\large \today}

\end{center}
\end{titlepage}
% \maketitle{}


\section*{Aims of the project}
% keep very short, just implementing an EBM model of PCA

We are planning on modelling PCA using an \textit{event-based model} (EBM), which models the progression of the disease as a sequence of events, where each event corresponds to a biomarker becoming abnormal. Such events could be: (1) a new area of the brain got affected, (2) the patient shows a drop in cognitive test scores. The EBM can calculate the most common progression of the events in the patient cohort and can be used to classify individual patients into the three main groups: cognitively normal, mild cognitive impairment and Alzheimer's disease. The EBM can also be used for longitudinal studies; for instance, it can stage patients after a predefined follow-up period.

\section*{Project Background}
% clinical challenges, PCA, why is the modelling challenging, other types of models.

Dementia and other related neurodegenerative disorders cost the UK approximatelly \pounds 26 billion a year. Developing a clear understanding of how thse disorders operate and progress over time could help us find more effective treatment. This project is investigating the progression of Posterior Cortical Atrophy (PCA), which is an early onset variant of Alzheimer's disease that  causes atropy of the posterior part of the cerebral cortex, resulting in disruption of the visual system. 

\section*{Motivation}

The EBM has already been used to study familiar and sporadic Alzheimer's disease and Huntington's disease, and this project aims to apply these techniques in the study of PCA. We will be interested to find the most common event progression, use it to classify patients into several groups according to their disease stage. We will also do some longitudinal studies, such as finding the probability of a patient converting from cognitively normal to mild cognitive impairment or from mild cognitive impairment to PCA over time. If time permits, we can also investigate possible extensions of the EBM, such as allowing for multiple progression sequences (Alex, IPMI).


\section*{List of tasks \hfill Deadline}

\renewcommand\arraystretch{4.4} %\setlength\minrowclearance{2.4pt}
\newcommand\taskHeader[1]{\Large{\textbf{#1}}}

\subsection*{\noindent Data collection and processing \hfill 7 April}

\subsection*{\noindent Read literature and implement the EBM \hfill 23 April}

\subsection*{\noindent Initial results: \hfill  30 April} 
% Positional variance matrices and EBM stage histograms  for different groups  

\subsection*{\noindent Analyse follow-up data\hfill  7 May}
% , compute prob of remaining CN, MCI over time  

\subsection*{\noindent Build disease stage classifier\hfill  14 May}
% cut points, sensitivity and specificity  

\subsection*{\noindent Simoultaneous MCMC sampling \hfill 27 May}

\subsection*{\noindent Differential Diagnosis of PCA vs other AD subtypes \hfill 8 July}

\subsection*{\noindent Validation of the results\hfill  20 July}
% , comparison with other methods  

\subsection*{\noindent Write the Mres project report  \hfill  14 August}

% Temporal diagram hilighting the order in which areas of brain get affected by PCA
% (maybe, if relevant) Comparison of event sequences across left-right hemispheres?  \hfill  


\section*{Summary of progress to date}

I have so far experimented with the EBM by trying to reproduce some of the results of A. Young, 2012. We cannot start writing the EBM for the PCA yet, as we are still waiting to receive the dataset from the Dementia Research Center.


\end{document}
