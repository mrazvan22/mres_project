\documentclass[12pt,a4paper,oneside]{report}


\usepackage{amsmath,amssymb,calc,ifthen}

\usepackage{float}
%\usepackage{cancel}

\usepackage[table,usenames,dvipsnames]{xcolor} % for coloured cells in tables

\usepackage{tikz}

\usepackage{mathptmx}

\usepackage{hyperref}
\hypersetup{
    colorlinks,
    citecolor=black,
    filecolor=black,
    linkcolor=black,
    urlcolor=black
}

% Nice package for plotting graphs
% See excellent guide:
% http://www.tug.org/TUGboat/tb31-1/tb97wright-pgfplots.pdf
\usetikzlibrary{plotmarks}
\usepackage{amsmath,graphicx}
\usepackage{epstopdf}
\usepackage{caption}
\usepackage{subcaption}

% highlight - useful for TODOs and similar
\usepackage{color}
\newcommand{\hilight}[1]{\colorbox{yellow}{#1}}

\newcommand\ci{\perp\!\!\!\perp} % perpendicular sign
\newcommand*\rfrac[2]{{}^{#1}\!/_{#2}} % diagonal fraction

\usepackage{listings}



% margin size
\usepackage[margin=2cm]{geometry}

\tikzstyle{state}=[circle,thick,draw=black, align=center, minimum size=2.1cm,
inner sep=0]
\tikzstyle{vertex}=[circle,thick,draw=black]
\tikzstyle{terminal}=[rectangle,thick,draw=black]
\tikzstyle{edge} = [draw,thick]
\tikzstyle{lo} = [edge,dotted]
\tikzstyle{hi} = [edge]
\tikzstyle{trans} = [edge,->]


\begin{document}
\belowdisplayskip=12pt plus 3pt minus 9pt
\belowdisplayshortskip=7pt plus 3pt minus 4pt


To whom it may concern,\\

% Intro
My name is Razvan Valentin Marinescu and I am an MRes student in the CDT Program in Medical Imaging at UCL. I would be interested to take part in the MiniMD course in late June because it is a fantastic opportunity that would allow me to observe medical imaging in clinical practice and get hands-on experience from clinicians involved in patient care. My supervisor Prof. Daniel Alexander has informed me about this course and would be more than happy for me to take part in it.

% my project
My MRes project is on \emph{Automatic diagnosis of Alzheimer's disease subtypes using disease progression modelling}. For this project, I collaborate a lot with researchers from the Institute of Neurology at Queen Square, as I use their brain MRI datasets, cognitive test scores and protein biomarker data for my project. My secondary supervisor, Dr. Sebastian Crutch is also a neuropsychologist based at Queen Square. This MiniMD course would be very useful for me because it would help me understand more about the work they do there and how to better interpret and analyse their datasets. It would also be very useful to spend one week observing cardiovascular imaging and clinical practice at the Royal Free Hospital, where I can better understand how to translate the technologies I develop in my PhD project to the clinic.

% more about he course
Furthermore, this opportunity would also provide me access to unique clinical facilities and I can make important contacts with leading researchers and clinicians in their field. I would potentially be able to take part in multidisciplinary team meetings, clinical diagnostics, operating theatre sessions, ward rounds and state-of-the-art clinical research facilities. Royal Free Hospital is also a renowned teaching hospital that was rated "excellent" for the quality of services by the Healthcare Comission in 2009. Similarly, the Institute of Neurology is one of the best centers in Europe for neuroscience research. Therefore, these venues are ideal for the miniMD course, particularly relevant for those doing work in translational research. 

At the end of the course, we will also write a small report and give a presentation to the other CDT students describing our experience. In my opinion, this would also be a fine opportunity for us to practice our presentation skills, which will be very important in our future research careers, where we always need to communicate our research ideas to fellow scientists and to the general public. 

Given the multidisciplinary nature of my project, I believe that taking part in this course would be very beneficial, allowing me to get further insights into clinical practice. If I am offered the place, I will endeavour to make the most out of it. I look forward to hearing from you.\\\\

Yours sincerely,\\

Razvan Valentin Marinescu

\end{document}
